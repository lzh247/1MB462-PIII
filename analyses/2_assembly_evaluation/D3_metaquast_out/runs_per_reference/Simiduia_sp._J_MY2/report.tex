\documentclass[12pt,a4paper]{article}
\begin{document}
\begin{table}[ht]
\begin{center}
\caption{All statistics are based on contigs of size $\geq$ 500 bp, unless otherwise noted (e.g., "\# contigs ($\geq$ 0 bp)" and "Total length ($\geq$ 0 bp)" include all contigs).}
\begin{tabular}{|l*{1}{|r}|}
\hline
Assembly & final.contigs \\ \hline
\# contigs ($\geq$ 1000 bp) & 4 \\ \hline
\# contigs ($\geq$ 5000 bp) & 2 \\ \hline
\# contigs ($\geq$ 10000 bp) & 2 \\ \hline
\# contigs ($\geq$ 25000 bp) & 0 \\ \hline
\# contigs ($\geq$ 50000 bp) & 0 \\ \hline
Total length ($\geq$ 1000 bp) & 35307 \\ \hline
Total length ($\geq$ 5000 bp) & 32163 \\ \hline
Total length ($\geq$ 10000 bp) & 32163 \\ \hline
Total length ($\geq$ 25000 bp) & 0 \\ \hline
Total length ($\geq$ 50000 bp) & 0 \\ \hline
\# contigs & 12 \\ \hline
Largest contig & 17842 \\ \hline
Total length & 40921 \\ \hline
Reference length & 3739502 \\ \hline
GC (\%) & 39.46 \\ \hline
Reference GC (\%) & 53.19 \\ \hline
N50 & 14321 \\ \hline
N75 & 14321 \\ \hline
L50 & 2 \\ \hline
L75 & 2 \\ \hline
\# misassemblies & 0 \\ \hline
\# misassembled contigs & 0 \\ \hline
Misassembled contigs length & 0 \\ \hline
\# local misassemblies & 0 \\ \hline
\# scaffold gap ext. mis. & 0 \\ \hline
\# scaffold gap loc. mis. & 0 \\ \hline
\# unaligned mis. contigs & 0 \\ \hline
\# unaligned contigs & 1 + 7 part \\ \hline
Unaligned length & 37947 \\ \hline
Genome fraction (\%) & 0.025 \\ \hline
Duplication ratio & 3.171 \\ \hline
\# N's per 100 kbp & 0.00 \\ \hline
\# mismatches per 100 kbp & 4904.05 \\ \hline
\# indels per 100 kbp & 106.61 \\ \hline
Largest alignment & 247 \\ \hline
Total aligned length & 1356 \\ \hline
NGA50 & - \\ \hline
\end{tabular}
\end{center}
\end{table}
\end{document}
